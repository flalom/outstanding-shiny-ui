\PassOptionsToPackage{unicode=true}{hyperref} % options for packages loaded elsewhere
\PassOptionsToPackage{hyphens}{url}
%
\documentclass[]{book}
\usepackage{lmodern}
\usepackage{amssymb,amsmath}
\usepackage{ifxetex,ifluatex}
\usepackage{fixltx2e} % provides \textsubscript
\ifnum 0\ifxetex 1\fi\ifluatex 1\fi=0 % if pdftex
  \usepackage[T1]{fontenc}
  \usepackage[utf8]{inputenc}
  \usepackage{textcomp} % provides euro and other symbols
\else % if luatex or xelatex
  \usepackage{unicode-math}
  \defaultfontfeatures{Ligatures=TeX,Scale=MatchLowercase}
\fi
% use upquote if available, for straight quotes in verbatim environments
\IfFileExists{upquote.sty}{\usepackage{upquote}}{}
% use microtype if available
\IfFileExists{microtype.sty}{%
\usepackage[]{microtype}
\UseMicrotypeSet[protrusion]{basicmath} % disable protrusion for tt fonts
}{}
\IfFileExists{parskip.sty}{%
\usepackage{parskip}
}{% else
\setlength{\parindent}{0pt}
\setlength{\parskip}{6pt plus 2pt minus 1pt}
}
\usepackage{hyperref}
\hypersetup{
            pdftitle={Outstanding User Interfaces with Shiny},
            pdfauthor={David Granjon},
            pdfborder={0 0 0},
            breaklinks=true}
\urlstyle{same}  % don't use monospace font for urls
\usepackage{longtable,booktabs}
% Fix footnotes in tables (requires footnote package)
\IfFileExists{footnote.sty}{\usepackage{footnote}\makesavenoteenv{longtable}}{}
\usepackage{graphicx,grffile}
\makeatletter
\def\maxwidth{\ifdim\Gin@nat@width>\linewidth\linewidth\else\Gin@nat@width\fi}
\def\maxheight{\ifdim\Gin@nat@height>\textheight\textheight\else\Gin@nat@height\fi}
\makeatother
% Scale images if necessary, so that they will not overflow the page
% margins by default, and it is still possible to overwrite the defaults
% using explicit options in \includegraphics[width, height, ...]{}
\setkeys{Gin}{width=\maxwidth,height=\maxheight,keepaspectratio}
\setlength{\emergencystretch}{3em}  % prevent overfull lines
\providecommand{\tightlist}{%
  \setlength{\itemsep}{0pt}\setlength{\parskip}{0pt}}
\setcounter{secnumdepth}{5}
% Redefines (sub)paragraphs to behave more like sections
\ifx\paragraph\undefined\else
\let\oldparagraph\paragraph
\renewcommand{\paragraph}[1]{\oldparagraph{#1}\mbox{}}
\fi
\ifx\subparagraph\undefined\else
\let\oldsubparagraph\subparagraph
\renewcommand{\subparagraph}[1]{\oldsubparagraph{#1}\mbox{}}
\fi

% set default figure placement to htbp
\makeatletter
\def\fps@figure{htbp}
\makeatother

\usepackage{booktabs}
\usepackage{amsthm}
\makeatletter
\def\thm@space@setup{%
  \thm@preskip=8pt plus 2pt minus 4pt
  \thm@postskip=\thm@preskip
}
\makeatother
\usepackage[]{natbib}
\bibliographystyle{apalike}

\title{Outstanding User Interfaces with Shiny}
\author{David Granjon}
\date{2020-04-12}

\begin{document}
\maketitle

{
\setcounter{tocdepth}{1}
\tableofcontents
}
\hypertarget{prerequisites}{%
\chapter{Prerequisites}\label{prerequisites}}

This book requires to be familiar with the R Shiny framework. We recommand reading \ldots{}

\hypertarget{intro}{%
\chapter{Introduction}\label{intro}}

To DO

\hypertarget{part-survival-kit}{%
\part*{Survival Kit}\label{part-survival-kit}}
\addcontentsline{toc}{part}{Survival Kit}

\hypertarget{introduction}{%
\chapter*{Introduction}\label{introduction}}
\addcontentsline{toc}{chapter}{Introduction}

This part will give you basis in HTML, JavaScript to get started\ldots{}

\hypertarget{part-htmltools}{%
\part*{htmltools}\label{part-htmltools}}
\addcontentsline{toc}{part}{htmltools}

While building a custom html template, you will need to know more about the wonderful \href{https://github.com/rstudio/htmltools}{htmltools} developed by Winston Chang, member of the shiny core team. It has the same spirit as devtools, that is, making your web developer life easier. What follows does not have the pretention to be an exhaustive guide about this package. Yet, it will provide you yith the main tools to be more efficient.

\hypertarget{practice}{%
\chapter{Practice}\label{practice}}

In this chapter, you will learn how to build your own html templates taken from the web
and package them, so that they can be re-used at any time by anybody.

\hypertarget{selecting-a-good-template}{%
\section{Selecting a good template}\label{selecting-a-good-template}}

There exists tons of HTML templates over the web. However, only a few part will be suitable
for shiny, mainly because of what follows:

\begin{itemize}
\tightlist
\item
  shiny is built on top of \href{https://getbootstrap.com/docs/3.3/}{bootstrap 3} (HTML, CSS and Javascript framework), meaning that going for another framework might
  not be straightforward. However, shinymaterial and shiny.semantic are examples showing
  this can be possible.
\item
  shiny relies on \href{https://jquery.com}{jQuery} (currently v 1.12.4 for shiny, whereas the latest version is 3.3.1). Consequently, all templates based upon React, Vue and other Javascript framework will not be natively supported. Again, there exist some \href{https://github.com/alandipert/react-widget-demo/blob/master/app.R}{examples} for React with shiny and more generally,
  the \href{https://react-r.github.io/reactR/}{reactR} package developed by Kent Russell (\citet{timelyportfolio} on Twitter) and Alan Dipert from RStudio.
\end{itemize}

See \href{https://github.com/rstudio/shiny/tree/master/inst/www/shared}{the github repository} for more details about all dependencies related to the shiny package.

Therefore in the following, we will restict ourself to Bootstrap (3 and 4) together with jQuery. Don't be disapointed since there is still a lot to say.

\begin{quote}
Notes: As shiny depends on Bootstrap 3.3.7, we recommand the user who would like to
experiment Boostrap 4 features to be particularly careful about potential incompatibilies.
See a working example here with \href{https://github.com/RinteRface/bs4Dash}{bs4Dash}.
\end{quote}

A good source of \textbf{open source} HTML templates is \href{https://colorlib.com}{Colorlib} and \href{https://www.creative-tim.com/bootstrap-themes/free}{Creative Tim}. You might also buy your template, but forget about the packaging option, which would be illegal in this particular case, unless you have a legal agreement with the author (very unlikely however).

\bibliography{book.bib,packages.bib}

\end{document}
